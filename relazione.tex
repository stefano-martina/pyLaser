
\documentclass[a4paper,oneside]{article}
\usepackage[utf8]{inputenc}
\usepackage{lmodern}
\usepackage[T1]{fontenc}
\usepackage[italian]{babel}
\usepackage{xcolor}
\usepackage{graphicx}
\usepackage{listings}

\newcommand{\unicocode}[1]{\texttt{\bfseries #1}}
\newcommand{\codei}[1]{\unicocode{#1}}
%  ``\lstinline[language=Python,basicstyle=\small\ttfamily\bfseries,breaklines=true]|#1|''


\lstnewenvironment{code}
{
  \lstset{language=Python, basicstyle=\small\ttfamily\bfseries,
    breaklines=true, breakatwhitespace=false, frameround=fttt,
    frame=trBL, backgroundcolor=\color{yellow!20}}
}{}


\begin{document}
\section{Descrizione software}
Il linguaggio usato per lo svolgimento delle simulazioni \`e
\emph{Python}. I motivi per cui \`e stato scelto tale linguaggio sono:
\begin{itemize}
\item la versatilit\`a nello stile di programmazione, che permette di
  realizzare programmi di diverse dimensioni, funzionalità e stili
  (funzionale \emph{OOP}, etc\dots),
\item la folta comunity di sviluppatori mette a disposizione una ricca
  serie di pacchetti per il calcolo numerico, il disegno di grafi, e
  in generale tutte le funzioni necessarie per la realizzazione dei
  modelli.

In totale sono stati realizzati tre programmi corrispondenti ai tre
modelli proposti dal libro:
\begin{description}
\item[haken.py] implementa il modello più semplice di \emph{Haken},
\item[milonni.py] implementa il modello di \emph{Milonni} e
  \emph{Eberly},
\item[maxwell.py] implementa il modello dato dalle equazioni di
  \emph{Maxwell-Bloch}.
\end{description}

\end{itemize}
\subsection{haken.py}
La prima parte \`e l'\emph{import} dei pacchetti e la definizione
delle \emph{costanti} usate nel modello.
\begin{code}
import scipy.integrate
import numpy as np
import matplotlib.pyplot as plt
import matplotlib.animation as ani
from pylab import *

#constants
G = 2.      #gain coefficient
k = 100.    #photon loss rate
a = 10.     #atoms drop to ground rate
\end{code}

Dopo c'è la definizione dell'equazione differenziale che identifica il
modello di Haken, questa è costituita da una funzione
\codei{makeHaken} che accetta come parametri dei valori e ritorna una
funzione (definita con la lambda notazione) in \codei{n} e \codei{t}
che \`e:
$$
\dot{n} = 
$$

\end{document}